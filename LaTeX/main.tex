\documentclass{article}
\usepackage{background}
\usepackage{lipsum}
\usepackage[margin=6pt]{geometry} % Remove all margins
\pagestyle{empty} % Disable page numbering

\newlength\mylen
\setlength\mylen{\dimexpr\paperwidth/40\relax}

\SetBgScale{1}
\SetBgAngle{0}
\SetBgColor{blue!30}
\SetBgContents{\tikz{\draw[step=\mylen] (-.5\paperwidth,-.5\paperheight) grid (.5\paperwidth,.5\paperheight);}}

\begin{document}
Co oznaczają następujące terminy: lokalizacja, lateralizacja, odsłuch monauralny, binauralny, diotyczny i dychotyczny? \\
Lokalizacja to zdolność określenia położenia źródła dźwięku w przestrzeni. Lateralizacja odnosi się do percepcji dźwięku jako dochodzącego bardziej z jednej strony, co pozwala na rozróżnienie kierunku źródła dźwięku. Odsłuch monauralny polega na słuchaniu dźwięku jednym uchem, natomiast binauralny to słuchanie obydwoma uszami z wykorzystaniem różnic między nimi do lepszej percepcji przestrzennej. Odsłuch diotyczny to sytuacja, w której identyczny sygnał dźwiękowy jest prezentowany jednocześnie do obu uszu. Dychotyczny odsłuch oznacza, że różne sygnały dźwiękowe są jednocześnie prezentowane do każdego z uszu, co pozwala badać, jak mózg przetwarza konkurencyjne informacje słuchowe. \\
Na czym polega „eksternalizacja” obrazu słuchowego? \\
Eksternalizacja obrazu słuchowego polega na takim przetwarzaniu i prezentacji dźwięku, aby słuchacz odbierał go jako pochodzący z zewnętrznej przestrzeni, a nie z wnętrza swojej głowy. W praktyce oznacza to, że dźwięki są postrzegane jako znajdujące się wokół słuchacza, co zwiększa realizm i immersję doświadczenia słuchowego. Proces ten wykorzystuje wskazówki przestrzenne, takie jak różnice międzyuszowe czasu (ITD), różnice poziomu (ILD) oraz funkcje przenoszenia związane z głową (HRTF). Dzięki temu możliwe jest odtworzenie wrażeń przestrzennych podczas słuchania przez słuchawki lub systemy dźwięku przestrzennego. Eksternalizacja jest kluczowa w aplikacjach takich jak dźwięk binauralny, wirtualna rzeczywistość czy gry komputerowe, gdzie ważne jest precyzyjne umiejscowienie źródeł dźwięku w przestrzeni. \\
Co to jest międzyuszna różnica czasu i międzyuszna różnica natężenia? \\
Międzyuszna różnica czasu (ITD) i międzyuszna różnica natężenia (ILD) to dwa kluczowe mechanizmy, dzięki którym układ słuchowy lokalizuje źródła dźwięku w przestrzeni. **ITD** odnosi się do niewielkich różnic w czasie, w jakim dźwięk dociera do każdego ucha; gdy dźwięk pochodzi z jednej strony, dociera do bliższego ucha nieco wcześniej niż do dalszego, a mózg wykorzystuje te różnice do określenia kierunku dźwięku. **ILD** dotyczy różnic w natężeniu dźwięku docierającego do każdego ucha; głowa działa jako bariera akustyczna, powodując, że dźwięk jest głośniejszy w uchu bliższym źródła, zwłaszcza dla dźwięków o wysokiej częstotliwości, a mózg wykorzystuje te różnice natężenia do lokalizacji dźwięku. Wspólnie, ITD i ILD umożliwiają precyzyjną lokalizację dźwięków w płaszczyźnie poziomej, co jest kluczowe dla orientacji i interpretacji środowiska akustycznego. \\
Jakiego rzędu międzyuszna różnica czasu powoduje przesunięcie obrazu słuchowego całkowicie w jedną stronę (tak, że dźwięk odbierany jest z boku)? \\
Aby dźwięk był odbierany jako dochodzący całkowicie z boku, międzyuszna różnica czasu (ITD) musi osiągnąć swoją maksymalną wartość. Dla ludzi maksymalna ITD wynosi około **0,6 milisekundy** (600 mikrosekund). Taka różnica czasu występuje, gdy źródło dźwięku znajduje się bezpośrednio po jednej stronie głowy, pod kątem 90 stopni względem linii środkowej twarzy. Jest to czas potrzebny falom dźwiękowym na przebycie odległości między jednym uchem a drugim. \\
**Odpowiedź:** Około 0,6 milisekundy międzyusznej różnicy czasu powoduje przesunięcie obrazu słuchowego całkowicie w jedną stronę. \\
Jakiego rzędu międzyuszna różnica natężenia powoduje przesunięcie obrazu słuchowego całkowicie na jedną stronę (tak, że dźwięk odbierany jest z boku)? \\
A difference in intensity of approximately **10 decibels (dB)** between the ears can cause the auditory image to shift completely to one side, making the sound appear to come from that direction. This occurs because our auditory system uses interaural intensity differences (IIDs) to localize sounds, especially at higher frequencies where these differences are more noticeable. Even smaller intensity differences can influence perceived direction, but around 10 dB is typically needed to fully lateralize the sound. This means that a sound louder by about 10 dB in one ear compared to the other will be perceived as coming entirely from that side. \\
**Answer:** Około 10 dB różnicy natężenia między uszami całkowicie przesuwa obraz słuchowy na jedną stronę. \\
W jakich zakresach częstotliwości o lokalizacji decyduje międzyuszna różnica czasu lub międzyuszna różnica natężenia. Co powoduje taki przebieg tych zależności? \\
Międzyuszna różnica czasu (ITD) decyduje o lokalizacji dźwięku w **niskich częstotliwościach**, od około **20 Hz do 1500 Hz**. Międzyuszna różnica natężenia (IID lub ILD) jest kluczowa w **wysokich częstotliwościach**, powyżej około **1500 Hz**, szczególnie powyżej **3000 Hz**. Przy niskich częstotliwościach długie fale dźwiękowe omijają głowę bez znaczącego tłumienia, więc różnice natężenia są niewielkie, ale różnice czasu są zauważalne i wykorzystywane przez mózg do lokalizacji. Przy wysokich częstotliwościach krótkie fale są tłumione przez głowę, tworząc cień akustyczny, co powoduje znaczące różnice natężenia między uszami, podczas gdy różnice czasu są trudniejsze do zinterpretowania ze względu na szybką zmianę fazy. Dlatego w niskich częstotliwościach lokalizację determinuje ITD, a w wysokich częstotliwościach IID, co wynika z fizycznych właściwości fal dźwiękowych i interakcji z głową. \\
Co to jest tzw. stożek niepewności? \\
Stożek niepewności to termin używany w różnych dziedzinach, takich jak meteorologia i zarządzanie projektami, aby opisać zakres niepewności związany z prognozami lub szacunkami. W meteorologii reprezentuje na mapie przewidywaną trasę centrum huraganu lub cyklonu, gdzie szerokość stożka odzwierciedla rosnącą niepewność w miarę upływu czasu. W zarządzaniu projektami stożek niepewności ilustruje, że na początku projektu szacunki dotyczące kosztów, harmonogramu czy zakresu są mniej dokładne, a niepewność maleje w miarę postępu prac. Koncepcja ta pomaga w planowaniu i zarządzaniu ryzykiem, uwzględniając, że precyzja prognoz zwiększa się wraz z dostępem do większej ilości informacji. Stożek niepewności wizualizuje zatem redukcję niepewności w czasie, co pozwala na lepsze podejmowanie decyzji. \\
Ile (ogólnie) wynoszą kątowe progi różnicowe przesunięcia źródła dźwięku w płaszczyźnie poziomej? \\
Ogólnie kątowe progi różnicowe przesunięcia źródła dźwięku w płaszczyźnie poziomej wynoszą około **1 stopnia** dla dźwięków dochodzących z przodu (azymut 0 stopni). Jest to tzw. **minimalny słyszalny kąt** (ang. *Minimum Audible Angle*, MAA). Wartość ta może się zmieniać w zależności od częstotliwości dźwięku, natężenia, akustyki pomieszczenia i indywidualnych cech słuchu. Dla dźwięków dochodzących z innych kierunków zdolność do wykrywania różnic kątowych może być mniejsza, a progi różnicowe mogą wzrosnąć do kilku stopni. \\
Co to jest zjawisko dominacji (pierwszeństwa)? \\
Zjawisko dominacji, znane również jako efekt pierwszeństwa, polega na tym, że pierwszy bodziec lub informacja, którą odbieramy, ma większy wpływ na naszą percepcję lub decyzje niż kolejne bodźce. W akustyce oznacza to, że gdy dwa dźwięki o podobnej treści docierają do nas z niewielkim odstępem czasowym, nasz mózg preferuje pierwszy z nich przy określaniu lokalizacji źródła dźwięku. W psychologii poznawczej efekt pierwszeństwa odnosi się do tendencji do lepszego zapamiętywania lub przywiązywania większej wagi do informacji prezentowanych na początku listy lub sekwencji. To zjawisko wpływa na sposób przetwarzania informacji i może mieć znaczenie w kontekście uczenia się, komunikacji czy marketingu. Ogólnie rzecz biorąc, zjawisko dominacji podkreśla wpływ pierwszych doświadczeń lub informacji na naszą percepcję i reakcje. \\
W jakim zakresie opóźnień sygnału występuje zjawisko dominacji? \\
Zjawisko dominacji, znane również jako efekt precedencji, występuje w akustyce, gdy dwa identyczne sygnały dźwiękowe docierają do słuchacza z niewielkim opóźnieniem czasowym. Jeśli opóźnienie między nimi wynosi od około 1 ms do około 30 ms, słuchacz lokalizuje dźwięk w kierunku pierwszego sygnału. W tym zakresie drugi sygnał nie jest odbierany jako oddzielny dźwięk czy echo, lecz wpływa na percepcję kierunku dźwięku, wzmacniając wrażenie pochodzenia z pierwszego źródła. Po przekroczeniu tej granicy czasowej drugi sygnał zaczyna być postrzegany jako oddzielny dźwięk lub echo. \\
**Odpowiedź:** Zjawisko dominacji występuje przy opóźnieniach sygnału od około 1 ms do około 30 ms. \\
Na czym polega odmaskowanie dwuuszne (BMLD)? \\
**Odmaskowanie dwuuszne (BMLD - *Binaural Masking Level Difference*) polega na poprawie zdolności wykrywania sygnału dźwiękowego w szumie dzięki różnicom między sygnałami docierającymi do obu uszu.** W sytuacji, gdy sygnał i szum są identyczne w obu uszach, próg wykrywania sygnału jest wyższy. **Wprowadzenie różnic fazowych lub natężeniowych między sygnałami w uszach obniża ten próg, co ułatwia wykrycie sygnału w obecności szumu.** Zjawisko to wynika z centralnego przetwarzania w mózgu, który analizuje różnice między sygnałami z lewego i prawego ucha. **Odmaskowanie dwuuszne jest kluczowe dla lokalizacji dźwięku i poprawy rozumienia mowy w hałaśliwym otoczeniu.** \\
Podać typowe wartości odmaskowania dwuusznego. \\
Typowe wartości odmaskowania dwuusznego (MLD) to różnice w progu słyszenia sygnału w obecności szumu przy korzystaniu ze słuchu obuusznego. MLD jest największe dla niskich częstotliwości, osiągając około **15 dB** dla 500 Hz. Dla średnich częstotliwości, takich jak 1000 Hz, MLD wynosi około **10 dB**. W przypadku wyższych częstotliwości, powyżej 1500 Hz, MLD maleje do wartości często poniżej **5 dB**. Te wartości pokazują, że słuch obuuszny znacząco poprawia wykrywanie sygnałów w szumie na niskich i średnich częstotliwościach.

\end{document}